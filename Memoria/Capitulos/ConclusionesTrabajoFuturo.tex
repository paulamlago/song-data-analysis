\chapter{Conclusiones y Trabajo Futuro}
\label{cap:conclusiones}

Como hemos podido ver no hemos podido comprobar nuestra hipótesis inicial de que en España la mayoría de canciones iban a tener un sentimiento, por lo general alegre y positivo, pero al haber visto que aquellas canciones que están en inglés están, a priori, mejor clasificadas sabemos que tendremos que incluir un proceso de traducción al inglés o bien traducir el diccionario de sentimientos a otros idiomas. No solo eso, si no que además perdemos mucho contexto a la hora de analizar las palabras, ya que al analizar las palabras individualmente no podemos saber el significado total de una frase. Para ello también tendremos que implementar técnicas de procesamiento de textos mucho más avanzadas y que requieren de mucho más tiempo, esfuerzo e investigación, pero que, en nuestra opinión, es algo alcanzable.

Futuras aplicaciones de este trabajo de investigación pueden ser, por ejemplo buscar momentos a lo largo de la historia en los que el sentimiento de las canciones cambibie e intentar encontrar causas históricas a las que pueda estar asociado, por ejemplo guerras, crisis o cuando se gana un evento deportivo. 

Si encontrasemos alguna relación entre el sentimiento que desprende la música y eventos históricos tendríamos pues que plantearnos si es el arte el que imita a la vida o es la vida la que imita al arte.