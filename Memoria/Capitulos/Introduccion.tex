\chapter{Introducción}
\label{cap:introduccion}

El auge de la producción musical en los últimos años ha propiciado un aumento en el volumen de los datos, de los cuales se puede extraer una gran cantidad de información. Para el desarrollo de este proyecto nos interesará obtener el contexto social y cultural del país de procedencia del autor de cada canción en función de las letras de sus canciones. Para ello utilizaremos R, el cual se adecua a la perfección a la tarea que queremos realizar, ya que cuenta con una gran cantidad de librerías ya creadas (tanto por R como por la comunidad) que nos ayudarán en gran medida a la hora de análisis de textos, generación de gráficos y minería web. Esta última técnica será una parte importante de nuestro trabajo, ya que necesitaremos obtener información de fuentes externas como Wikipedia, para completar la información que nos aportan los datasets de los que partiremos.


En los siguientes capítulos plantearemos en detalle los objetivos propuestos al inicio del desarrollo del proyecto (Capítulo \ref{cap:Objetivos}). A continuación expondremos la metodología utilizada en el Capítulo \ref{cap:metodologia} para después explicar en detalle los pasos seguidos en la Implementación del proyecto en el Capítulo \ref{cap:implementacion}. Los resultados obtenidos tras implementar el sistema se explican en el Capítulo \ref{cap:resultadosyevaluacion}. Finalmente las conclusiones y el trabajo futuro se presenta en el último Capítulo, número \ref{cap:conclusiones}.